% \documentclass[a4paper, 12pt]{article}
\documentclass[conference]{IEEEtran}

% geometry
\usepackage[top=0.5in]{geometry}
\usepackage[english]{babel}
% \usepackage{blindtext}

\usepackage[
    %backend=biber, 
    natbib=true,
    style=numeric,
    sorting=none
]{biblatex}
\addbibresource{export.bib}

\begin{document}
\title{\Large{\textbf{The usefulness of photogrammetry in digitizing Rwanda's cultural assets}}}
\author{By Jean Nshuti \\ jnshuti@andrew.cmu.edu \\ Carnegie Mellon University — School of Engineering}
\date{March 23, 2022}
\maketitle

\textbf{\textit{Abstract} — The research investigates the necessity to have cultural tools digitized by inspecting existing methods
    and technologies used in the sector of assets digitization. The research will spectate the effect of cultural assets' destruction on the society
    and course of actions used to prevent that. The paper then focuses on practical considerations and presenting an overview of the technicalities using photogrammetry. \\}

\textbf{\textit{Keywords — cultural assets, cultural heritage artifacts, photogrammetry, digitization}}

\section{\textbf{Introduction}}
The use of photogrammetry has been on the rise since its inception some 150 years ago \cite{histphtgm}. It has been adopted by a wide range of users in fields
like research, engineering, architecture, archaeology, geology, etc.
The  American Society for Photogrammetry and Remote Sensing defined photogrammetry as \cite{Ebert2015} “the art, science, and technology
of obtaining reliable information about physical objects and the environment, through processes of recording, measuring,
and interpreting imagery and digital representations of energy patterns derived from noncontact sensor systems” \\

The very rich historical background of Rwanda can be attributed to the large number of cultural assets she possesses. These assets are considered valuable and
delicate since damage of any form devalues their worth and makes it difficult to replicate or reconstruct. Total annihilation of these assets in case of misfortunes
could likely erase of trace of their existence from history. As such, digitizing such assets will preserve their existence even if they end up being destroyed. \\

A technology that captures qualitative information like texture, color and refractive index of assets has to be used. This technology will ensure that features
of these assets are capture to a higher degree of precision before being stored digitally \cite{Linder2006}. This will help when
the asset is being stored digitally to reduce the margin of error and increase accuracy \cite{accphtgm}. With this, photogrammetry can be used to further
investigate how useful it would be in digitizing Rwanda's cultural assets. \\

Since photogrammetry's foundation is based on pictures, they will be taken, processed and then digitized to produce 3D objects/assets that are cleaned and retopologized in
a 3D application. \\

As technology advances, photogrammetry becomes more user-friendly. So, using a reconstruction software
would be more effective with an expectation of reducing the loss due to machine learning. In addition, using a DSLR (digital single-lens reflex) camera would produce assets of
high resolution. Gathering several pictures of different assets from museums will be the quickest way to obtain data. By comparing the digitized
asset and the real (physical) asset, we will be able to analyze and have a viable conclusion. This will result in accurate and clean assets that will
hold every data needed in digitization. \\

Hence, this research aims at investigating the need to preserve cultural heritage and suggesting proved methods and ways to conduct a digitization of cultural heritage using
photogrammetry in Rwanda

\section{\textbf{Literature Review}}
Cultural assets (heritage materials) are as essential as the culture itself, as they hold the meaning and history of a place.
It is the physical and intangible characteristics of civilization that have been passed down through the centuries \cite{WILLIS2014145}. For cultural assets hold a heritage
and are irreplaceable, their demise would create a gap in one nation's history hence impossible to pass the culture down.
So, the need to digitally preserve/conserve the heritage removes the unbearable effects. \\

An unexpected destruction to any of the artifacts would not only result in cultural deficiency but in an economic deficit as well. For instance, in 2016, New Delhi’s
National Museum of Natural History was caught on fire where a 160 million old fossils and staffed animals were annihilated \cite{delhi012}. For that reason, digital preservation should
be of immense interest to the nation of Rwanda. \\

\subsection{\textbf{The History of cultural heritage digitization}}
The act of cultural assets protection/conservation started in the early 18th century, where an Austrian ruler called Maria Theresa set regulations about their protection in times of war \cite{germ72}.
But the term "cultural heritage" was adopted in November 1972 in a General Conference of UNESCO \cite{batisse01}, which indicated the benefits of protecting the unique, invaluable assets to everyone.
In the 19th \& 20th centuries, scholars felt a need to create collections of their researches after the WWII, which institutions used attract more scholars \cite{Note2011}. It was not until October 2003 when UNESCO developed a
"Charter on the Preservation of Digital Heritage", which held principles it needed to follow in order to preserve digital heritage of various assets such as
books, monuments, etc., thus the beginning and need to preserve and share the knowledge \cite{charter002}. \\

\subsection{\textbf{Methods used in cultural heritage digitization}}
The basis of cultural heritage digitization is on 2D and 3D archiving, which results in either 2D photographs or 3D models. The earliest an dsimplest form of digitization was images taken with care and stored in
museum galleries. For example, in the 1940s, The Museum of Modern Art inaugurate its photography department and George Eastman House International Museum of Photography and Film was
opened for this matter \cite{Note2011}. But as technology advances, 2 methods were invented, namely, photogrammetry and laser-scanning \cite{dgpht}. Photogrammetry which uses an \textit{image-based modelling} approach has a merit of
generating quick and effective data processing to the building of an asset. For instance, the method was utilized in China to restore  Emperor Qin’s mausoleum \cite{Zhou2012}. Companionably, laser-scanning uses \textit{range-based modelling}
which permits for the acquisition of numerous minor geometric information of an asset hence providing better precision \cite{dgpht} but with a drawback of slow data capturing, and need a multitude of manpower and resource \cite{Zhou2012}.
Regardless, the latter in parallel with other technologies was used to scan Tang Paradise \cite{dgpht}.

\subsection{\textbf{The impact of cultural heritage digitization}}
The eradication of heritage in any form disempowers any nation's heritage. Access to the digital heritage would create more chances for people to create, communicate and share knowledge \cite{charter002}. Additionally, virtual tours
which is the use of 360\textsuperscript{0} videos or VR (virtual reality) to tour/explore a certain place virtually, can benefit from these assets and facilitate the sector of tourism. Notably, in the 1990s, Williams and Hobson, predicted
that tourism was about to enter the age of VR and be used to promote tourism \cite{Yang2021}. Hence, the digitized heritage would immensely support that shift in Rwanda, in case applied. Additionally, digital education with cultural heritage
would be facilitated by the digitized assets. For instance, in the Netherlands, during the COVID-19 pandemic \cite{eddig}, teachers, cultural heritage professionals, researchers worked with Europeana and European Schoolnet to create
digital heritage to better integrate it in education. This end result was that 17,000 students and more benefited from this and in the proved to augment abilities in the areas of cognition, social interaction, and culture.  With the help of
Rwanda Cultural Heritage Agency, this technology could not only aid students but commoners as well, in case the physical access to assets is restricted due an uncertainty. \\

\section{\textbf{Methodology}}
\subsection{Approach}
This research paper attempts to investigate the necessity to cultural digitize heritage and suggest an effective way to carry that task using photogrammetry in Rwanda. 

\section{\textbf{Conclusion}}
The review investigated the history of cultural heritage digitization, methods, and the impact of digitizing cultural heritage by assessing what has been done before.
The research showed how the need to preserve cultural assets started as early as the 18th century from an Austrian ruler during war. Even though, digitizing cultural assets started with images as soon as photographs were invented, later two
advanced, effective methods namely laser-scanning and photogrammetry were utilized. Digitized cultural heritage impacts are several but tourism and education were found to be more closely impactful to the society. From this research it's clear
that Rwanda as a regional ICT/Tech hub \cite{rwtechub} would leverage this technology to preserve and conserve its culture for next generations using photogrammetry. Moving forward, this research recommends effective technicalities driven past literatures on how
cultural heritage can be preserved in Rwanda.

% \begin{enumerate}
%     \item one
%     \item two
% \end{enumerate}



\printbibliography

\end{document}

